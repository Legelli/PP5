%==================================================
%      PREAMBOLO e DICHIARAZIONI INIZIALI
%==================================================
\documentclass[10pt,oneside,a4paper]{article}

\usepackage[latin1]{inputenc} 
\usepackage[italian]{babel}
\usepackage[T1]{fontenc}
\usepackage{siunitx} %Inserisce automaticamente i dati con le unit�  di misura correttamente formattate del SI (utilizzo: \SI{0.82}{m^2}, in generale \SI{misura con il punto decimale}{unit�  di misura})
\sisetup{output-decimal-marker = {.}, separate-uncertainty = true, input-uncertainty-signs = \pm, detect-weight=true, detect-family=true} %per usare SI con il punto decimale
\usepackage{listings} %Per citare codice informatico formattandolo correttamente
\usepackage{amsmath,amsthm,verbatim,amssymb,amsfonts,amscd,graphicx,mathtools}
\usepackage[makeroom]{cancel}
\newcommand{\abs}[1]{\left\lvert\,#1\,\right\rvert}
\usepackage{geometry}
\usepackage{epigraph}
\usepackage{booktabs}	%tabelle migliorate
\usepackage{tablefootnote}	%note a pi� di pagina in tabella
\usepackage{threeparttable} %tabella con note a pi� di tabella
\usepackage{caption}	%descrizione per figure
\usepackage{dblfnote}
\captionsetup{tableposition=top,figureposition=bottom,font=small} %setup descrizione
\usepackage{float}
\usepackage{esvect} %vettori
\usepackage{longtable} %tabelle lunghe
\usepackage[dvipsnames]{xcolor}
\definecolor{sepia}{HTML}{80002A}
\usepackage[colorlinks=true, citecolor=black, linkcolor=sepia, urlcolor=black]{hyperref}
\usepackage{mathrsfs}
\usepackage{circuitikz}
\ctikzset{bipoles/resistor/height=0.2}
\ctikzset{bipoles/resistor/width=0.4}
\usepackage{enumitem} %Liste senza spazi verticali
\setlist{noitemsep}
\usepackage{amsmath}


\usepackage{multicol}
\newenvironment{Figure}
  {\par\medskip\noindent\minipage{\linewidth}}
  {\endminipage\par\medskip}

\newcommand{\var}{\operatorname{var}}
\newcommand{\cov}{\operatorname{cov}}


\usepackage{listings} %Per inserire codice
\lstnewenvironment{codice_c}[1][]
{\lstset{basicstyle=\small\ttfamily, columns=fullflexible,
keywordstyle=\color{red}\bfseries, commentstyle=\color{blue},
language=C, basicstyle=\small,
numbers=left, numberstyle=\tiny,
stepnumber=2, numbersep=5pt, frame=shadowbox,  showstringspaces=false, #1}}{}

%==================================================
%                  PRIMA PAGINA
%==================================================

\title{\textsc{\textbf{Esercitazione 5}: Amplificatore operazionale 3}}
\author{\small{G. Galbato Muscio} \and \small{L. Gravina} \and \small{L. Graziotto}}
\date{20 Novembre 2018}

\begin{document}
	\begin{figure}
		\centering
		\includegraphics[scale=0.5, trim={2.8cm 8.9cm 0 9cm}, clip]{logo.png}
	\end{figure}
	\maketitle
	\begin{center} 
		\fbox{{\fontsize{12pt}{8mm}\textsc{Gruppo 11}}} \\
	\end{center}
\hrule
\vspace{0.5cm}
\renewcommand{\abstractname}{Abstract}
\begin{abstract}
Si realizzano un filtro VCVS passa basso e un generatore di rumore bianco basato su transistor di tipo 2N2222A; si studia l'andamento del modulo e della fase della funzione di trasferimento del filtro passa basso al variare della frequenza del segnale in ingresso.
\end{abstract}
\vspace{4cm}
\tableofcontents %Indice
\newpage


\pagebreak
\begin{multicols}{2}
%==================================================
%      		FILTRO VCVS
%==================================================
\section{Filtro VCVS passa basso}
Si utilizza nel seguito l'amplificatore operazionale LM358N, e lo si alimenta con una differenza di potenziale di $\pm \SI{15}{V}$, connettendo gli ingressi \texttt{V+} e \texttt{GND} alla doppia alimentazione in continua, positiva e negativa. 

Si realizza il seguente filtro passa basso VCVS.

%ESEMPIO DI CIRCUITO
\begin{center}
\begin{circuitikz}[scale=0.7]
\draw (0,0) node[op amp, yscale=-1] (opamp) {}
(opamp.+) -- ++(-1.2,0) node[circ]{} to[C=$C_1$] ++(0,-1.5) node[ground]{}
(opamp.out) to[short, *-*] ++(0.5,0) node[right] {$v_\text{o}$}
(opamp.-) -- ++(0,-1) node[circ] {} to[R=$R_4$] ++(0,-1.5) node[ground]{}
(opamp.-) ++(0,-1) to[R=$R_3$] ++(3,0) -| (opamp.out)
(opamp.+) ++(-1.2,0) to[R=$R_2$] ++(-2,0) to[R=$R_1$] ++(-2,0) node[circ]{} node[left]{$v_i$}
(opamp.+) ++(-1.2,0) ++(-2,0) node[circ]{} -- ++(0,1.5) to[C=$C_2$] ++(4,0) -| (opamp.out)
;\end{circuitikz}
\end{center}

Nell'ipotesi di amplificatore ideale (quindi corrente d'ingresso nulla e tensioni ai due ingressi uguali) la funzione di trasferimento assume la forma 
\tiny{
\begin{equation}\label{eq:trasferimento} %IL PRIMO MODO PER RIPORTARE UN'EQUAZIONE LUNGA � SPEZZARLA
	\text{T}(\omega) = \frac{k}{1- \text{R}_1 \text{R}_2 \omega^2 \text{C}_1 \text{C}_2 + j \omega^2(\text{C}_1(\text{R}_1 + \text{R}_2) + \text{R}_1 \text{C}_2 (1-k))}
\end{equation}}
\normalsize
che per $\text{R}_2 = \text{R}_1$ e $\text{C}_2 = \text{C}_1$ si semplifica in
\begin{equation}\label{eq:trasferimento_approx}
	\text{T}(\omega) = \frac{k}{1 - (\omega \text{R}_1 \text{C}_1)^2 + j \omega \text{R}_1 \text{C}_1 (3-k)}
\end{equation}
%Nel caso generale, modulo e fase della funzione di trasferimento sono rispettivamente
%\begin{equation}\label{eq:modulo_trasf}%IL SECONDO MODO � SCALARLA
%	\resizebox{0.5\textwidth}{!}{$\vert \text{T} (\omega) \vert = \frac{\text{k}}{\sqrt{\left( 1 - \text{R}_1 \text{R}_2 \omega^2 \text{C}_1 \text{C}_2 \right)^2 + \omega^2 \left(\text{C}_1 (\text{R}_1 + \text{R}_2) + \text{R}_1 \text{C}_2 (1-\text{k})\right)^2}}$}
%\end{equation}
%\begin{equation}\label{eq:fase_trasf}
%	\phi(\omega) = \text{arctan}\left(\frac{\omega(\text{C}_1(\text{R}_1+\text{R}_2)+\text{R}_2\text{C}_2(1-\text{k}))}{1-\text{R}_1\text{R}_2\omega^2\text{C}_1\text{C}_2}\right)
%\end{equation}
Il modulo e la fase della funzione di trasferimento valgono dunque
\small{
\[
\lvert T(w)\rvert = \frac{k}{\big(1-\omega^2(C_1R_1)^2\big)^2+\big((3-k)^2\omega^2C_1^2R_1^2\big)^2}
\]}
\small{
\[
\phi(\omega)= -\arctan\left(\frac{(3-k)\omega C_1R_1}{1-\omega^2(C_1R_1)^2}\right)
\]}
\normalsize


Si vuole scegliere una resistenza $\text{R}_1$ sufficientemente grande (nell'ordine delle decine di \SI{}{\kilo\ohm}) per aumentare la resistenza di ingresso $\text{R}_\text{i}$ del filtro e dunque ridurre le perturbazioni del segnale in ingresso.

Per ottenere $\text{R}_1$ il pi� possibile simile a $\text{R}_2$ si � usato per la seconda un trimmer, un altro trimmer � stato usato per la resistenza $\text{R}_3$ in modo da poter controllare con precisione il valore di amplificazione del filtro.

Il circuito � stato realizzato con i seguenti componenti fissi:
\begin{itemize} %Lo spazio verticale � stato tolto con \setlist{noitemsep} nell'header, rimuovere quella riga per avere una lista classica
	\item[--] $\text{R}_1 = \SI{111 \pm 111}{\kilo\ohm}$;
	\item[--] $\text{R}_2 = \SI{111 \pm 111}{\kilo\ohm}$;
	\item[--] $\text{R}_4 = \SI{111 \pm 111}{\kilo\ohm}$;
	\item[--] $\text{C}_1 = \SI{111 \pm 111}{\nano\farad}$;
	\item[--] $\text{C}_2 = \SI{111 \pm 111}{\nano\farad}$
\end{itemize}
Il valore di $\text{R}_3$ non � riportato tra gli altri non avendo esso un valore fisso.

Si stima la frequenza di taglio del circuito, dati i componenti usati, dalla~(\ref{eq:trasferimento}):
\[
f_T = \frac{1}{2\pi\sqrt{R_1R_2C_1C_2}} = \SI{111 \pm 111}{Hz}.
\]

\subsection{Overshooting}

Come prima misura, si sceglie il valore di 
\[
R_3 = (k-1) R_4 = \SI{111 \pm 111}{\kilo\ohm},
\] ovvero si sceglie $k = \SI{111}{}$: essendo $k > l_\text{limite} \equiv 1.586$ ci si aspetta un picco del modulo della funzione di trasferimento in corrispondenza della frequenza di taglio del filtro. 

I risultati delle misure sono riportati in Tabella~\ref{tab:filtro_uno}. Dalle misure di tensione a frequenze sufficientemente basse (si considerano le prime 111 misure) si pu� ricavare un valore di amplificazione reale pari a $A_v = \SI{-111 \pm 111}{}$ con incertezza data dalla deviazione standard; il valore misurato risulta compatibile con quello previsto, ossia $A_v^\text{\footnotesize{(atteso)}} = k$. Gli andamenti del modulo e della fase della funzione di trasferimento sono riportati in Figura~\ref{fig:modulo_filtro} e in Figura~\ref{fig:fase_filtro}.%, sovrapposte ai dati sperimentali ci sono le curve teoriche ottenute da (\ref{eq:modulo_trasf}) e (\ref{eq:fase_trasf}) corrispondenti ai valori misurati dei componenti.

\begin{center}
\captionof{table}{Misure dello studio in frequenza del filtro VCVS con $\text{k} >  \text{k}_\text{limite}$.}
\label{tab:filtro_uno}
\begin{tabular}{c|c|c|c}
$f$ [\SI{}{Hz}] & $v_i$ [\SI{}{V}] & $v_o$ [\SI{}{mV}] & $\Delta t [\SI{}{s}]$ \\
\hline
\hline
\end{tabular}
\end{center}

%\begin{Figure}
%	\begin{center}
%	\includegraphics[width=\linewidth]{MODULO FUNZIONE DI TRASFERIMENTO}
%	\captionof{figure}{Diagramma di Bode del modulo della funzione di trasferimento al variare delle frequenze. Si riportano gli andamenti per un filtro di Butterworth e per un valore di k superiore a quello di quest'ultimo, sovrapposte ai dati sperimentali si riportano le curve teoriche. }
%	\label{fig:modulo_filtro}
%	\end{center}
%\end{Figure}

%\begin{Figure}
%	\begin{center}
%	\includegraphics[width=\linewidth]{FASE FUNZIONE DI TRASFERIMENTO}
%	\captionof{figure}{Diagramma di Bode della fase della funzione di trasferimento al variare delle frequenze. Si riportano gli andamenti per un filtro di Butterworth e per un valore di k superiore a quello di quest'ultimo, sovrapposte ai dati sperimentali si riportano le curve teoriche. }
%	\label{fig:fase_filtro}
%	\end{center}
%\end{Figure}

Gli andamenti sono compatibili con quelli teorici: il modulo ha un massimo in corrispondenza di $f_T = \SI{111 \pm 111}{\hertz}$, valore che � stato stimato come la frequenza intermedia tra le misure corrispondenti ai due valori di $\vert T(f) \vert$ pi� elevati, compatibile con quello ricavabile da (\ref{eq:trasferimento_approx}) pari a $f_T^{\text{\footnotesize{(attesa)}}} = \SI{111 \pm 111}{\hertz}$; inoltre la fase assume il valore di $-\pi / 2$ in corrispondenza di $f_\phi = \SI{111 \pm 111}{\hertz}$ inferito dalla retta congiungente i due punti sperimentali pi� vicini al valore teorico assunto della fase alla frequenza di taglio, anche questo valore risulta compatibile con quello teorico.

\subsection{Butterworth}

Come seconda misura si sceglie il valore di $\text{R}_3 = \SI{111 \pm 111}{\kilo\ohm}$, ovvero si sceglie $\text{k} = \SI{111} \approx \text{k}_\text{limite}$, cio� si realizza un filtro \emph{Butterworth}. In queste condizioni non ci si aspetta alcun picco del modulo della funzione di trasferimento. I risultati delle misure sono riportati in Tabella~\ref{tab:filtro_due}, gli andamenti del modulo e della fase della funzione di trasferimento sono riportati in Figura~\ref{fig:modulo_filtro} e in Figura~\ref{fig:fase_filtro}.

\begin{center}
\captionof{table}{Misure dello studio in frequenza del filtro VCVS con $\text{k} = \text{k}_\text{limite}$}
\label{tab:filtro_due}
\begin{tabular}{c|c|c|c}
$f$ [\SI{}{Hz}] & $V_i$ [\SI{}{V}] & $v_o$ [\SI{}{mV}] & $\Delta t [\SI{}{s}]$ \\
\hline
\hline
\end{tabular}
\end{center}

Gli andamenti sono ancora compatibili con quelli teorici, infatti dall'andamento del modulo si stima una frequenza di taglio pari a $\omega_{\text{T}} = \SI{111 \pm 111}{\hertz}$ ottenuto come la frequenza dell'intersezione tra l'interpolazione lineare in regime costante (misure da 1 a 111) traslata verso il basso di $\SI{-3}{\decibel}$\footnote{Ossia il valore del modulo della funzione di trasferimento alla frequenza di taglio.} e quella a regime di discesa asintotica (misure da 111 a 111), valore compatibile con quello teorico di $\omega_\text{teo} = \SI{111 \pm 111}{\hertz}$; anche dall'andamento della fase si pu� stimare, come fatto nel paragrafo precedente, la frequenza di taglio ottenendo $\omega_\phi = \SI{111 \pm 111}{\hertz}$ ancora compatibile con l'aspettativa teorica.

%==================================================
%      		GENERATORE DI RUMORE
%==================================================
\section{Generatore di rumore}
Si realizza un generatore di rumore \emph{bianco} amplificando quello generato da una giunzione pn in regime di breakdown, il circuito costruito � il seguente
\begin{center}
\begin{circuitikz}
\draw (0,0) node[op amp] (opamp) {}
(opamp.+) to[R=$R_1'$] ++(-2,0) node[circ] {} node[left] {$v_\text{2}$}
(opamp.-) to[short] ++(0, 1.25) to[R = $R_2$] ++(2,0) -| (opamp.out)
(opamp.-) node[circ] {}
(opamp.+) node[circ] {}
(opamp.-) to[R=$R_1$] ++(-2,0) node[circ] {} node[left] {$v_1$}
(opamp.out) to[short, *-*] ++(0.5,0) node[right] {$v_\text{o}$}
(opamp.+) to[R=$R_2'$] ++(0,-2) node[ground] {}
;\end{circuitikz}
\end{center}
dove i componenti scelti hanno valori:
\begin{itemize} %Lo spazio verticale � stato tolto con \setlist{noitemsep} nell'header, rimuovere quella riga per avere una lista classica
	\item[--] $\text{V}_\text{in} = \SI{111 \pm 111}{\volt}$;
	\item[--] $\text{R}_1 = \SI{111 \pm 111}{\kilo\ohm}$;
	\item[--] $\text{R}_2 = \SI{111 \pm 111}{\kilo\ohm}$;
	\item[--] $\text{R}_3 = \SI{111 \pm 111}{\kilo\ohm}$;
	\item[--] $\text{R}_4 = \SI{111 \pm 111}{\kilo\ohm}$;
	\item[--] $\text{C}_1 = \SI{111 \pm 111}{\nano\farad}$;
	\item[--] $\text{C}_2 = \SI{111 \pm 111}{\nano\farad}$
	\item[--] $\text{C}_3 = \SI{111 \pm 111}{\nano\farad}$
	\item[--] $\text{C}_4 = \SI{111 \pm 111}{\nano\farad}$
\end{itemize}

Il collettore del transistor $\text{Q}_1$ � lasciato flottante in quanto lo si vuole utilizzare come diodo, entrambi i transistor sono di tipo 2N2222A.

Si riporta in Figura~\ref{fig:rumore} il segnale di rumore rilevato dall'oscilloscopio, il rumore ha ampiezza picco-picco misurata con l'oscilloscopio pari a $\text{V}_\text{noise} = \SI{111 \pm 111}{\milli\volt}$.
Collegando l'uscita del generatore all'ingresso del filtro passa basso costruito alla sezione precedente si ottiene il segnale riportato in Figura~\ref{fig:rumore_filtro}: questo rumore risulta attutito rispetto al caso precedente in quanto le componenti del rumore a frequenza pi� alta vengono abbattute dal filtro.

%\begin{Figure}
%	\begin{center}
%	\includegraphics[width=\linewidth]{RUMORE SEMPLICE}
%	\captionof{figure}{Istantanea dell'oscilloscopio del segnale di rumore bianco prodotto da una giunzione pn in regime di breakdown.}
%	\label{fig:rumore}
%	\end{center}
%\end{Figure}

%\begin{Figure}
%	\begin{center}
%	\includegraphics[width=\linewidth]{RUMORE FILTRO}
%	\captionof{figure}{Istantanea dell'oscilloscopio del segnale di rumore bianco attutito da un filtro passa basso VCVS}
%	\label{fig:rumore_filtro}
%	\end{center}
%\end{Figure}


%ESEMPIO DI FIGURA
%\begin{Figure}
%	\begin{center}
%	\includegraphics[width=\linewidth]{comune.png}
%	\captionof{figure}{Istantanea dell'oscilloscopio per l'amplificatore differenziale, misura di $A_c$}
%	\label{fig:Ac_differenziale}
%	\end{center}
%\end{Figure}


%ESEMPIO DI TABELLA
%\begin{center}
%\captionof{table}{Misure per la stima di $A_c$}
%\label{tab:Ac_differenziale}
%\begin{tabular}{c|c|c|c}
%$f$ [\SI{}{Hz}] & $V_i$ [\SI{}{V}] & $v_o$ [\SI{}{mV}] & $A_c = v_o / V_i$ \\
%\hline
%      149.5 &        3.90 &         11.3 & 2.90e-03 \\
%      222.0 &        3.90 &         11.5 & 2.95e-03 \\
%      281.0 &        3.90 &         11.8 & 3.03e-03 \\
%      359.0 &        3.90 &         11.8 & 3.03e-03 \\
%      461.0 &        3.90 &         12.2 & 3.13e-03 \\
%\hline
%\end{tabular}
%\end{center}
\end{multicols}
\end{document}